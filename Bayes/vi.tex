\documentclass{article}

% if you need to pass options to natbib, use, e.g.:
%     \PassOptionsToPackage{numbers, compress}{natbib}
% before loading neurips_2018

% ready for submission
% \usepackage{neurips_2018}

% to compile a preprint version, e.g., for submission to arXiv, add add the
% [preprint] option:
%     \usepackage[preprint]{neurips_2018}

% to compile a camera-ready version, add the [final] option, e.g.:
     \usepackage[final]{neurips_2018}

% to avoid loading the natbib package, add option nonatbib:
%     \usepackage[nonatbib]{neurips_2018}

\usepackage[utf8]{inputenc} % allow utf-8 input
\usepackage[T1]{fontenc}    % use 8-bit T1 fonts
\usepackage{hyperref}       % hyperlinks
\usepackage{url}            % simple URL typesetting
\usepackage{booktabs}       % professional-quality tables
\usepackage{amsfonts}       % blackboard math symbols
\usepackage{nicefrac}       % compact symbols for 1/2, etc.
\usepackage{microtype}      % microtypography
\usepackage{amsmath}
\usepackage{amssymb}

\title{Variational Inference}

% The \author macro works with any number of authors. There are two commands
% used to separate the names and addresses of multiple authors: \And and \AND.
%
% Using \And between authors leaves it to LaTeX to determine where to break the
% lines. Using \AND forces a line break at that point. So, if LaTeX puts 3 of 4
% authors names on the first line, and the last on the second line, try using
% \AND instead of \And before the third author name.

\author{%
  David S.~Hippocampus\thanks{}\\%Use footnote for providing further information
    %about author (webpage, alternative address)---\emph{not} for acknowledging
    %funding agencies.} \\
  Department of Computer Science\\
  Cranberry-Lemon University\\
  Pittsburgh, PA 15213 \\
  \texttt{hippo@cs.cranberry-lemon.edu} \\
  % examples of more authors
  % \And
  % Coauthor \\
  % Affiliation \\
  % Address \\
  % \texttt{email} \\
  % \AND
  % Coauthor \\
  % Affiliation \\
  % Address \\
  % \texttt{email} \\
  % \And
  % Coauthor \\
  % Affiliation \\
  % Address \\
  % \texttt{email} \\
  % \And
  % Coauthor \\
  % Affiliation \\
  % Address \\
  % \texttt{email} \\
}

\begin{document}
% \nipsfinalcopy is no longer used

\maketitle

\begin{abstract}
VI, VB, EM Summary
\end{abstract}

\section{Summary}
Given $\theta$ as parameter, $x$ observed, $z$ latent variables, we have
\begin{equation}
l(\theta; D) = \sum_z\log p(z|\theta) p(x|z, \theta)
\end{equation}
According to Jensen's Inequality (log is concave), we have
\begin{eqnarray}
l(\theta; D) =\log\sum_z q(z|x) \frac{p(x|\theta)}{q(z|x)}\\
\ge \sum_z q(z|x)  \log\frac{p(x|\theta)}{q(z|x)}
\end{eqnarray}
The lower-bound is called free energy. The equality satisfies when $q(z|x) = p(z|x,\theta)$

\subsection{Application 1: GMM}
\begin{equation}
\log p(\theta; D)\ge \sum_z q(z|x)  \log\frac{p(x|\theta)}{q(z|x)}
\end{equation}
E-step:
\begin{equation}
q^{t+1} = \arg\max_q F(q, \theta^t)
\end{equation}
M-step:
\begin{equation}
\theta^{t+1} = \arg\max_{\theta} F(q^{t+1}, \theta)
\end{equation}

\subsection{Application 2: VAE}
Encoder: $z = q(\phi,x)$, decoder: $x=p(\theta,z)$. We have MLE:
\begin{eqnarray}
KL(q(z|x), p(z|x)) & = & E_{q(z|x)}\log q(z|x) -  E_{q(z|x)}(\log P(x|z) + \log p(z) - \log p(x)) \\
E_{q(z|x)}\log p(x) &=& E_q \log p(x|z) - KL(q(z|x), p(z)) + KL(q(z|x), p(z|x))\\
 &=& ELBO + KL(q(z|x), p(z|x))
\end{eqnarray}
and loss to optimize is:
\begin{equation}
l(\phi, \theta) = -E_{x\sim q(x|z)} \log p(x|z) + KL(q(z|x), p(z))
\end{equation}

\section{Fisher Information Matrix, Natural Gradient}
KL-Divergence:
\begin{eqnarray*}
&KL(p_w(x), p_{w+\triangle w}(x)) \\
=& E_{x\sim p_w(x)} \log p_w(x) - \log p_{w+\triangle w}(x)\\
=& E_{x\sim p_w(x)} \{\log p_w(x) - [\log p_w(x)+\nabla_w \log p_w(x)\triangle w + \frac{1}{2}\triangle w^T \nabla^2_w \log p_w(x)\triangle w)]\} \\
=& [E_{x\sim p_w(x)}\nabla_w \log p_w(x)]\triangle w -  \frac{1}{2}\triangle w^T [E_{x\sim p(x)}\nabla^2_w \log p_w(x)]\triangle w\\
= & \frac{1}{2}\triangle w[E_{x\sim p(x)}\nabla_w \log p_w(x)\nabla_w \log p_w(x)^T]\triangle w^T
\end{eqnarray*}
where
\begin{eqnarray*}
\nabla^2_w \log p_w(x) = \frac{\nabla^2_w p_w(x)}{p_w(x)} - \frac{\nabla_w p_w(x)\nabla_w p_w(x)^T}{p_w^2(x)} \\
= \frac{\nabla^2_w p_w(x)}{p_w(x)} - \nabla_w \log p_w(x)\nabla_w \log p_w(x)^T
\end{eqnarray*}

\end{document}
